\documentclass{llncs}

% UTF8 support
\usepackage[utf8x]{inputenc}
\usepackage{url}
\usepackage{graphicx}
\usepackage{float}

\usepackage{paralist}

\usepackage[draft]{fixme}
\usepackage{todonotes}

\graphicspath{{figs/}}
\newcommand{\eg}{{\textit{e.g.,~}}}
\newcommand{\etal}{{\textit{et al.~}}}
\newcommand{\ie}{{\textit{i.e.~}}}

\begin{document}

\title{Simulation and HRI\\Recent Perspectives with the MORSE Simulator}

\author{
Séverin Lemaignan\inst{1} \and 
Marc Hanheide\inst{2} \and 
Michael Karg\inst{3} \and 
Harmish Khambhaita\inst{4} \and \\
Lars Kunze\inst{5} \and 
Florian Lier\inst{6} \and 
Grégoire Milliez\inst{4} \and 
Ingo Lütkebohle\inst{7}}

\authorrunning{Séverin Lemaignan et al.}

\institute{CHILI Lab, EPFL, Lausanne, Switzerland
\and Centre for Autonomous Systems, University of Lincoln, United Kingdom
\and IAS, Technische Universität München, Germany
\and LAAS/CNRS, Université de Toulouse, France
\and Intelligent Robotics Lab, University of Birmingham, United Kingdom
\and CITEC, Bielefeld University, Germany
\and Machine Learning and Robotics Lab, Universität Stuttgart, Germany}



\maketitle

\begin{abstract}

Simulation in robotics is often a love-hate relationship: while simulators do
save us a lot of time and effort compared to regular deployment of complex
software architectures on complex hardware, simulators are also known to evade
many of the real issues that robots need to manage when they enter
the real world.  Because humans are the paragon of dynamic, unpredictable,
complex, real world entities, simulation of human-robot interactions may look
condemn to fail, or, in the best case, to be mostly useless. This collective
article reports on five independent applications of the MORSE simulator in the
field of human-robot interaction: It appears that simulation is already useful,
if not essential, to successfully carry out research in the field of HRI, and
sometimes in scenarios we do not anticipate.

\end{abstract}

\section{Introduction}

The use of simulators for human-robot interaction (HRI) encompasses a variety of
use-cases, from prototyping through evaluation to anticipatory simulation at
runtime. It, however, suffers from a specific integration problem: Simulation in
HRI requires to model robots in all their complexity \emph{plus} a means of
representing and interacting with human agents. We therefore believe that an
important stepping stone for a wider use of simulation in HRI is the
availability of an integrated, easy-to-use framework that can encompass all
currently important use-cases, and that provides an integration interface for
developers \emph{and} end-users of HRI simulation.  In particular, we feel that
it must be both easy to install and use, and offer adequate domain abstractions
to facilitate development and integration. This paper presents how recent work
using the \emph{Modular OpenRobots Simulation Engine}~\cite{morse_simpar_2012}
(MORSE, figure~\ref{fig|morse-hri}) attempts to address this challenge. 

We will first review the range of current use-cases for simulation in HRI, then
introduce MORSE with a focus on its HRI specific features, and finally
demonstrate and discuss MORSE's versatility through several case studies. The
case studies also illustrate the collective nature of this article: We report on
contributions and experiences in human-robot interaction simulation from five
unrelated projects, conducted by different people in different organizations,
only sharing the MORSE simulator as common simulation platform. 

%The sections~\ref{sc:assessment} to~\ref{sc:ci}
%present each of these projects, and try to highlight both the positive
%outcomes of deploying simulation environments for HRI, and the pitfalls and more
%fundamental issues that simulation of human-robot interaction still faces.

\subsection*{HRI and simulation}

\begin{figure}[t]
      \centering 
      \includegraphics[width=0.7\linewidth]{morse_pr2.jpg}
      \caption{Simulation and HRI: A PR2 and a human avatar in MORSE.}
      \label{fig|morse-hri}
\end{figure}

\subsubsection*{Applications of simulation in HRI}

\begin{inparaenum}[\itshape 1\upshape)]
In the HRI literature, several distinct goals for the use of simulation can be 
discerned. Without claiming completeness, we categorize them into \item prototyping, 
\item human modeling, \item interactive simulation, and \item anticipatory simulation.  
\end{inparaenum}

The most well-known use-case is probably \emph{prototyping}: The use of a
simulator to reconstruct and run experiments in a simulated target situation
prior to real-world evaluation. Apart from convenience, reasons to do so include
simulation of unsafe situations (\eg navigation in narrow
spaces~\cite{sisbot2007human,kidokoro2013will} or
crowds~\cite{henry2010learning}), and exploration of edge cases (\eg humans not
paying attention~\cite{knepper2012pedestrian,guzzi2013human}).

\emph{Human modeling} is one way of realizing human agents in simulation.
\cite{garrell2010model} present for example a pedestrian model which has been
evaluated against a large-scale database of recorded human movements. When
detailed motion or other actions (such as speaking) is required, cognitive
models have been used in many areas~\cite{sun2006cognition}, and increasingly
also in HRI (\eg ~\cite{trafton2013act}).  It is probably safe to say that such models are
still far from general, but already quite useful for specific situations.

A problem with modeling is the significant up-front effort required. Therefore, 
some research has explored the use of game engines in what
we call \emph{interactive simulation}: a real human controls a simulated human
avatar interactively. While not fully automated, it allows reliable capture of
interaction data for later analysis. This has been used for a long time in
tele-operation settings~\cite{wang2005validating} and also more recently for
so-called ``crowd-sourcing'' work~\cite{breazeal2013crowdsourcing}.

The last use case, which is becoming more popular recently, is a very different 
use of simulation: \emph{anticipatory simulation}. Its goal
is to choose amongst a range of possible actions by comparing multiple possibilities
in simulation, immediately prior to execution on a physical robot. This
makes use of, \eg the physics and spatial computation subsystems in simulators,
to \emph{anticipate} likely outcomes. 
Examples of such applications are
to compute social metrics such as walking comfort~\cite{kidokoro2013will} or
proxemics~\cite{hoffman2010effects}.

We believe it is clear that these use-cases benefit from each other. Particularly
prototyping requires models, which could be manually specified, learned from 
real-world data, or learned interactively through the simulator. Anticipatory
simulation can make use of all these, and, additionally, requires flexible software
able to run faster than real-time.

\subsubsection*{Simulators for HRI}

Softwares used for HRI simulation are currently fairly diverse, and can be
distinguished by their use-cases. Prototyping work often uses
``standard'' robot simulators such as USARSim~\cite{lewis2007usarsim}
(commonly used for rescue robotics applications but also beyond), or
Gazebo~\cite{Koenig2004} (though Gazebo's human agent support
is currently limited) and MORSE~\cite{morse_simpar_2012}. 

In contrast, work in models or more advanced use-cases such as interactive or
anticipatory simulation currently uses custom software -- this is true for all
of the papers cited in the previous sections at least. The pedestrian modeling
community seems to share some tools, \eg the work by Treuille et 
al~\cite{treuille2006continuum} is known to have been re-used, but it has no 
connection to robot simulators.

Both standard robot simulators and pedestrian simulators use fairly coarse 
human models. In contrast, work in the Embodied Virtual Agent (EVA) community
usually provides higher-level functionality, such as simulated emotion dynamics, behavior
generation based on action primitives, conversational dialogue systems, and up
to cognitive simulations. However, integrating these into a coherent system with
an acceptable interface remains challenging~\cite{gratch2002creating}.

As stated before, we think that the integration of these diverse functionalities
into standard robot simulators would be an ideal next step, making specialized
tools available to a much wider audience, and thus likely also identifying new
avenues for improvements. We further believe that Open Source software, such as
MORSE or Gazebo, is particularly able to support this, because everything is open
and easily changed. This contrasts with otherwise capable frameworks, such as
USARSim, based on proprietary engines.

\subsection*{HRI and the MORSE simulator}

All five projects that are presented in this article rely on the
domain-independent MORSE simulator as simulation platform. MORSE is an
open-source tool developed for academic robotic research with contributions from
over 15 institutions worldwide. It extends the Blender \emph{Game Engine}, a 3D
engine which features shader-based 3D rendering and physics simulation ({\sc
Bullet} physics engine). This allows for semi-realistic simulation of complex
environments. The MORSE components (sensors and actuators) exchange data with
the robotics software via middleware bindings (\emph{Software In The Loop}
architecture).  Four middlewares designed for robotics are currently supported,
including ROS and YARP, as well as a generic socket-based protocol. This design
aims at providing a seamless experience when switching back and forth between
the simulator and the physical robot. Standard robotic platforms, actuators and
sensors (more than 50 components) are provided and enable fast creation of
simulation scenarios, while custom components and behaviors can be added via
simple Python scripts.

%Two MORSE features stand out. First, MORSE has been primarily designed with a
%command-line interface, and only features a minimal (and fully optional)
%graphical user interface. This makes MORSE mainly targeted to an academic
%audience, where efficiency and lightness prevail.  Simulation scenes are
%actually short Python programs, thus well suited for sharing and versioning.
%This also eases the integration of the simulator into larger development
%workflows, and MORSE is successfully integrated into several continuous
%integration systems (Travis, Jenkins).

MORSE also introduces a concept of \emph{abstraction levels}: sensors and actuators
may expose several levels of abstraction, corresponding to different level of
realism. For instance, users may choose if the odometry sensor returns only a
curvilinear distance, a $dX, dY, dZ$ differential vector, or the absolute
position of the robot (integrated odometry). This allows users that are testing
low-level components to do so, while users working at higher abstraction
levels (typically in HRI) do not have to run full robotic software stacks (and
thus, benefit from a lighter environment) and can work in a more deterministic
environment. This feature can be finely controlled, on a per-component basis.

For HRI applications, MORSE provides a human avatar that can be fully controlled
(displacement, gaze, grasping of objects, interactions with the environment like
turning lights on, opening drawers, doors...) from a first-person-shooter perspective.
This enables the researcher to quickly setup and test human-robot interactions
with a tele-operated human model, hence with realistic human behaviors. As
presented in~\cite{lemaignan2012morse}, the human avatar can be controlled using
a Kinect-like device. The same avatar can also be programmatically controlled
by external scripts, like any robot in MORSE. With standard MORSE actuators like
the \emph{waypoint} actuator, the researcher can, for instance, either pre-define paths
that the human avatar will follow in a simulated environment, or plug in a 
pedestrian model.

\section{HRI Simulation : Five Scenarios}

Both to demonstrate how versatile MORSE is, and to further illustrate 
how simulation can support research in HRI, we now present five case-studies.  
The first three scenarios, \emph{Situation
Assessment for HRI and Simulated Feedback}, \emph{An Expectations Framework for
Domestic Robot Assistants} and \emph{Preliminary Testing of Human-Aware
Navigation Planner} illustrate the typical approach for simulation: rather
complex virtual environments are created where human presence plays a
central role, and HRI algorithms are tested in a convenient and repeatable way.
Note that, while we introduce here \emph{simulation-only} scenarios, they all
are prototypes of experiments that have been conducted on real robots:
simulation is used here to prepare for real-world deployments.

The fourth scenario, \emph{Data Acquisition through Automatic Scene Generation}
shows how simulation is used as an alternative source of input to train robots to
behave in human environments, and the last scenario, \emph{Automated Execution
of Prototype HRI Experiments}, presents how the simulator can be used to provide
automatic testing of human-aware behaviors, fully integrated in the software
development workflow. Each of the presentations follow the same structure: we
first introduce the scenario, then highlight how simulation has been leveraged
and its benefits, and finally mention some of the shortcomings of the tool.

\subsection{Situation Assessment for HRI and Simulated Feedback}
\label{sc:assessment}

For human-robot interaction, understanding the environment in which
agents interact is a key issue. In particular, \emph{perspective taking}
is important to understand human actions and act appropriately. 
This use-case uses MORSE as a virtual environment in which to apply situation 
assessment algorithms (in our case, the SPAtial Reasoning and Knowledge
(SPARK) package~\cite{Milliez2014}). The robot updates its knowledge using its 
own position, human position and objects seen through abstracted, symbolic cameras 
provided by MORSE (so-called \emph{semantic} cameras). For example, consider 
figure~\ref{fig|spark}, where the human is sitting on a couch and asks the robot to
bring specific objects, which may be in another room (Fetch-and-Carry task).

\begin{figure}[t]
      \centering
      \includegraphics[width=0.7\linewidth]{morsespark.png}
      \caption{On the left side, the MORSE environment ; on the right side, the same
      environment, as perceived by the robot in the SPARK situation assessment
      module.}
      \label{fig|spark}
\end{figure}

\textbf{Benefits of the simulation} The primary functional benefit of MORSE here 
is for \emph{anticipatory simulation}. Moreover, individual algorithms can also be measured 
in a repeatable environment (that is, without possible confounding effects due variability
of real humans). 
Moreover, relying on MORSE effectively supports collaboration between the partners 
involved in this project (MaRDi project\footnote{\url{http://mardi.metz.supelec.fr}}): 
our partners are also
using MORSE simulation to \emph{prototype} their software and collect data with the same
environment in their laboratory, where they focus on dialog processing. They can
train their dialog system using MORSE feedback to test the robot behaviors~\cite{simparmardi2014},.

\subsection{An Expectations Framework for Domestic Robot Assistants}
\label{sc:expectations}

This case study considers domestic service robots, \ie a robot is
co-located in a human appartment, carrying out various services. As part of 
this, a goal is to detect novel situations, both for future improvement, and 
for possible self-adaptation. It does so by monitoring a set of
expectations~\cite{Karg2013}.
 
\textbf{Benefits of the simulation} Developing a monitoring system requires
re-testing a large number of both expected und unexpected behaviors for
every change. Doing so manually for every step during development would be 
prohibitively expensive. In contrast, using a simulator enabled us to
set up and run a large testbed of easy-to-generate unpredictable human behaviors,
using the human component of MORSE. The use of the simulated
scenario also enables us to gain many insights into the problem domain in a scenario
that would not been possible within our project. Of course, the algorithms were
eventually validated on the real robot, inside a smaller real-world environment.

\begin{figure}[t]
      \centering
      \includegraphics[width=0.7\linewidth]{morse_apartment.png}
      \caption{A simulated apartment with a domestic service robot and a person.}
      \label{fig|apartment}
\end{figure}

The human component of MORSE enabled us to test and validate our approach
dynamically in a variety of situations. Moreover, since it can be controlled
\emph{interactively}, like in a 3D computer game, it is easily possible to generate 
a multitude of situations to which the robot has to react. This greatly supported 
our project to gain insights about our approach, detect weak points and make improvements.

\subsection{Preliminary Testing of Human-Aware Navigation Planner}
\label{sc:navigation}

To \emph{prototype} and evaluate improvements in the human-aware navigation planner 
developed at LAAS, user studies are usually first carried out in simulation,
then in the real-world. For example, we set up an experiment,
where a robot encounters a human crossing its path (at $90^{\circ }$ angle to
each other) while the robot is moving forward to its navigation goal. MORSE
was first used to model the environment, using the included modeler, and then
for running the simulated environment, based on the included PR2 simulation. For
more details, see~\cite{ThibaultKruse2014}).


\textbf{Benefits of the simulation} Development of human-robot interaction
algorithms often require iterative process of prototyping, testing and
reviewing. Setting up and experiment and testing of robot navigation algorithms
especially for large environments involving humans is time consuming and is
subject to availability of lab resources while working in a shared lab between
different groups of researchers. Full support of the PR2 robot model among
others, availability of a human model, and a convenient way of setting up
experiment environment using Blender software were the most prominent features
for choosing MORSE as the simulation environment for these experiments. Since
MORSE already provides ROS bindings for the PR2 robot and human pose it requires
minimal effort to switch between real-world and simulated environments.

As a consortium member in the EU project
SPENCER\footnote{\url{http://www.spencer.eu}}, we plan to develop novel
algorithms for robot navigation in large populated environments, \eg airports.
In the future we plan to use MORSE to simulate such large environment with
multiple human models. This will certainly push the limits of simulation for HRI
and hopefully provide new benchmarks.

\subsection{Data Acquisition through Automatic Scene Generation}
\label{sc:generation}

This fourth study proposes a different perspective on the role of simulation in
HRI: simulating credible human environments to train systems to appropriately
react to them: autonomous mobile robots that are to help and assist people in
care homes, households, and at other workplaces have to understand how human
activities affect the dynamics of objects in the environment. That is, robots
need to know, when, where and how people manipulate objects and how they arrange
and structure them in space. In the context of the STRANDS
project\footnote{\url{http://www.strands-project.eu}} we aim for robots that
understand the long-term, spatio-temporal relationships of objects and
activities of people. In the scenario described here, we looked in
particular at learning qualitative spatial relations of objects on office desks.
As an accurate classification and pose estimation of objects on real-world
office desks is still a challenging and difficult task for current robot
perception systems we acquired a data set of object arrangements using the MORSE
simulator. For this, we first bootstrapped an object statistics from manually
labeled images of real office desks, and secondly, automatically generated a set
of physically possible desktop scenes
(figure~\ref{fig:simulated-desktop-scenes}). Based on the generated data we
learned relational models of object arrangements on desks. The learnt models
enabled a robot to predict the position of an object given a landmark. We
employed these models to effectively guide a simulated and a real robot in
object search tasks and evaluated its performance~\cite{kunze14indirect}.

\textbf{Benefits of the simulation} First, the automatic scene generation (made
easy by the use of Python to ``program'' the simulation scenes) and annotation
of object arrangements in simulation is useful for the acquisition of large
amounts of data over short periods of time. The generated data enabled us to
design, implement and to evaluate our methods for predicting object locations
before having a real-world data set in place. Secondly, the generation of object
arrangements can increase the variability of scenes in human-robot experiments
in general. Given the dynamics of objects in the real world it is important not
to oversimplify human-robot experiments in simulated environments but make them
as realistic as possible (in a controlled way). Finally, in future work, we plan
to use the generated desktop scenes in web-applications to crowdsource Natural
Language descriptions of object arrangements and commands for robots from
Internet users.

\begin{figure}[t]
  \centering
  \includegraphics[width=.7\columnwidth]{figs/scenes.png}
  \caption{Automatically generated scenes of office desks.}
  \label{fig:simulated-desktop-scenes}
\end{figure}


\subsection{Automated Execution of Prototype HRI Experiments}
\label{sc:ci}

In human-robot interaction studies, robots often indicate behavioral variability
that may influence the experiment's final outcome.  However, manual testing on
physical systems is usually the only way to prevent this, but remains
labour-intensive. To tackle this issue, we introduced \emph{early automated
prototype testing}~\cite{2645922} that consists of: a simulation environment, a
software framework for automated bootstrapping of prototype systems, execution
verification of system components, automated result assessment of experiments
and a Continuous Integration (CI) server to centralize experiment execution. In
our setup we bootstrap and execute a simulated prototype system on a CI server
and assess the results in each run. In this particular scenario, a robot must
report the location of a virtual human in a domestic environment. Both the
robot and the human are moving in the scene and meet in front of a table.


The goal of this simulation setup is to incrementally decrease the level of
abstraction until a satisfactory/sufficient degree of ``realism'' to make an
assumption about real world behavior, is reached --- in an integrated and
continuous approach. In order to achieve this goal, we make use of two essential
MORSE features: \textit{a)} the human avatar that can be steered (set waypoints)
interactively via middleware and \textit{b)} a \emph{semantic} camera that
extracts the location of a specific entity in the simulation environment. The
semantic camera is attached to the robot. If the human enters the robot's field
of view, the location is reported and sent via middleware. After each CI run,
the recorded movement trajectory of the human avatar is assessed (plotted) and
archived. We have explicitly chosen to simplify the extraction of the location
of the human to acquire a ground truth in the first iterations of the
simulation. As an example, a system component (running outside of the simulation) 
that is intended to classify whether there is a human in front of a robot, by fusing 
multiple sensory inputs, can be evaluated based on this ground truth. Subsequently, 
we are able to exchange/add diverse virtual sensors, \ie add a simulated laser 
scanner to build a person hypothesis for instance, thus gradually develop, 
assess and implement more complex scenarios.

\textbf{Benefits of the simulation} First of all, the interactive (remotely
controllable) human avatar is useful to include a dynamic, yet not too
realistic, human component in this setup. Secondly, the level of abstraction of different
sensors, \ie semantic camera versus virtual laser scanner enables us to
gradually raise the level of complexity/realism and test different algorithms
based on abstract and almost realistic sensor inputs. Lastly, the chance to
deploy MORSE in a Continuous Integration environment, \ie automatically run
simulation scripts, generates an additional benefit.

\section{Discussion: Towards Unification}

While the five scenarios that we present here implement different use-cases,
they actually cover similar approaches, while relying on the same simulator:
study~\ref{sc:assessment} shows how MORSE can be thought as a computation
engine, \ref{sc:expectations} exploits the human agent in a computer game style,
\ref{sc:navigation} uses MORSE for assessing and tuning the performance of
algorithms, and scenario~\ref{sc:generation}, while somewhat unique, still share
similarities with Garrell et al., in that a model for object positions is
trained on real-world data. Finally, the use-case presented in~\ref{sc:ci}
proposes a different approach, with a focus on continuous testing, and can
arguably be seen as the natural progression of using simulators for evaluation,
extended here to cover HRI scenarios.

From this perspective, one may consider that the experiments recently conducted
in the MORSE community around the simulation of HRI applications constitute the
first steps towards building an unifying platform for HRI simulation, with two
additional features: its \emph{programmability} (simulation scenarios are Python
scripts) and its concept of \emph{abstraction levels} that provides an effective
way to focus simulation on a particular problem by hiding irrelevant simulation
artifacts.

These diverse use-cases support the idea that simulation is not only actually
useful as a support tool for development of human-robot applications, but also
\emph{enables} new research techniques in HRI. \emph{Continuous Integration}
illustrates this point: while HRI experiments are considered as notoriously
difficult to deploy, test and repeat, we show here how a simulator may enable
automated testing of more and more complex scenarios, including long-term
interaction.

Several issues are also raised and must be clearly stated. In its current state,
the MORSE simulator provides only an incomplete model of the environment.
Sounds/speech models are incomplete, and human models do not yet provide good
enough accuracy, both at the level of the user interface (some actions can not
be done with the interactive avatar), and at the simulation level (poor/missing
walking cycles for instance). Finally, the overall convenience of MORSE for HRI
could be improved, for instance by providing more assets (furnitures, objects)
related to human environments. These issues, that are mostly technical and could
be addressed at the software level, show that simulators dedicated to HRI
application still need to mature. In that regard, the next section presents some
of the directions that are currently researched.

\subsection*{The next steps}

Several noteworthy developments related to HRI are currently shaping up in the
MORSE community. We outline below some of them, that suggest new applications
that we believe are relevant to HRI research.

A first line of investigation relates to the procedural generation of a variety
of realistic human models. {\sc MakeHuman} is such an open-source tool that
generates anatomically, kinetically and visually realistic human models. This
software has a tight integration with Blender, and MORSE is soon to provide as
well seamless integration with {\sc MakeHuman} models. This will bring a wide
range of characters to feed the simulations, and extend testing environments
with gender/size/age/skin color variances.

Besides being able to control a human avatar in simulation programmatically and
deterministically, the possibility to automatically generate believable and
realistic crowd behaviors is being explored. In this context, the objective is
to adopt in MORSE technologies previously developed for computer games to generate
trajectories that control the MORSE avatars. Based on the idea of \emph{social
forces}, the work of~\cite{Szymanezyk2012crowd} is to be
adapted to provide believable and realistic movement of several humans within
MORSE. This
implementation would provide a more realistic and dynamic environment to study
human-robot spatial interaction and to provide a testbed for human-aware motion
planning, to give two exemplary use-cases.

Another line of investigation looks at \emph{embedding} the researcher into the
robotic simulation. The purpose of such efforts is to provide a life-like
immersive simulation environment that would allow at the same time ecologically
valid human behaviors and repeatable, lightweight interaction settings.
In~\cite{lemaignan2012morse}, we presented how a human agent could interact with
a virtual robot through a deictic interface based on a Kinect. Two distinct
projects are currently looking into extending this approach, one (at Bielefeld
University) aiming at integrating emerging Virtual Reality devices (like Occulus
Rift) with MORSE, the other one (MarDI project) developing a virtual reality
cave, that include 360° projections and spatialized sound.

Also often suggested, the \emph{on-line} deployment of HRI simulations could
efficiently support large scale HRI studies. The simulator and specific
interaction scenarios would be embedded in a dedicated webpage and users would
control a human avatar from their webbrowsers. This would potentially enable
collection of large behavioral datasets. While MORSE development in that
direction has yet to start, Breazeal et al. presented an initial attempt in that
direction in~\cite{breazeal2013crowdsourcing} and the Gazebo simulator features
a limited WebGL client that act as a proof-of-concept of on-line robotic
simulation.

%Faster-than-realtime simulation, a recent feature of MORSE (currently at testing
%stage), is a last development that can have significant applications for
%HRI\fixme{finish that}

\subsection*{Conclusion}

These examples and ideas hopefully give a picture of the lively landscape of the
``Simulation for HRI'' community, that has built itself around the MORSE
simulator.  In the introduction, we mentioned how simulation in HRI had to
address in parallel constraints stemming from \emph{robotic simulation} and
\emph{virtual agent simulation}, while remaining a lightweight, easy-to-use
tool. We are certainly not yet there, much remains to be imagined, refined and
achieved.  Yet MORSE is already deployed in several institutions as a platform
that efficiently supports research in human-robot interaction. As an open-source
project, we strive for new use-cases and ideas, and warmly welcome researchers
that would like to join the effort.


\section*{Acknowledgment}

This research has received funding from the European
Union (FP7/2007-2013) under grant agreements
FP7-600623 (STRANDS) and FP7-600877 (SPENCER).

\bibliographystyle{abbrv}
\bibliography{main}

\end{document}
